% !TeX root = ./18.01.2021.tex
\documentclass{meetingmins}

\setdate{18th January 2021}
\setcommittee{SECAMP Team meeting}

\setpresent{
    Dennis Knoop,
    Jason Paul,
    Manish Katoch,
    Tejveer Walia,
    Rohan Sanap
}

\begin{document}
\maketitle

\section{Announcements}
    \begin{items}
        \item Weekly meetings for the time being, members should prepare a short presentation each week.
    \end{items}

\section{Updates}
    \begin{items}
        \item All - The team met on 15.01.2021 to discuss pending work and to pool questions. 
    \end{items}

\section{Discussion}
    \subsection{Telescope assembly}
    \begin{subitems}
        \item Adjustment and alignment mechanisms could be improved.
        \item Threaded track makes it hard to adjust the focal point exactly. Tejveer suggested to replace the current design with a threadless design with integrated vernier scale and catch screw. A CAD concept will be prepared for the next meeting.
        \item System for aligning the angular orientation of the lenses makes use of 3 screws in a track assembly. Dennis suggested that commercial solutions exist which enable the use of a single screw.
        \item The current telescope baseplate design is semi rectangular, it would be good to make it properly circular.
        \item Hole for the photo diode should be included in the new design to enable it's inclusion at a later date.
        \item Can look up fibre collimators online for more ideas about the design.   
    \end{subitems}
    \subsection{Vacuum Chamber}
    \begin{subitems}
        \item The new chamber CAD the team was given is MAIUS heritage and does not include the upper and lower halves.
        \item Upper and lower halves will include windows and ports.
        \item The upper and lower halves will be isolated from the central vacuum chamber, and will not be depressurised.
        \item Central chamber contains 8 ports, overall design will require around 13/14 ports. 
            Need to define this number based on the requirements of the overall project. 
            For the meantime develop the concept with a given number of ports and adjust the design if needed later.
        \item Begin designing the upper and lower halves for the meeting next week.
    \end{subitems}
    \subsection{Coils}
    \begin{subitems}
        \item Look into Dennis's Masters thesis for information regarding the sizing of the coils. 
        \item magnetic field requirement is somewhere around 10 gauss per cm. 
        \item Can research Anti-helmholz coils / Maxwell coils for more information.
    \end{subitems}
\section{Action Items}
    \begin{items}
        \item Team will develop a CAD concept for the telescope for the next meeting.
        \item Team will develop a CAD concept for the Vacuum chamber upper and lower hemispheres for the next meeting.
        \item Team members should put together a short summary of their work to present at the next meeting.
    \end{items}

\nextmeeting{25.01.2021 at 12:30}

\end{document}